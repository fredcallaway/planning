
\section{Introduction}
\label{sec:introduction}

%\paragraph{Big picture: rational analysis, cognitive strategies, rational process models, resource-rational analysis}
%\fl{Todo: Falk and Fred}
Previous research has shown that many aspects of human cognition can be understood as rational adaptations to the environment and the goals people pursue in it \citep{Anderson1990,Chater1999}. \textit{Rational analysis} leverages this assumption to derive models of human behavior from the structure of the environment. In doing so, rational analysis makes only minimal assumptions about cognitive constraints. %As a consequence, the resulting models are often more abstract than the kind of process models that are commonly developed in psychology.
However, it has been argued that there are many cases where the constraints imposed by cognitive limitations are rather substantial, and Herbert Simon famously argued that to understand people's cognitive strategies we have to  consider both the structure of the environment and cognitive constraints simultaneously \citep{Simon1956,Simon1982}. \textit{Resource-rational analysis} \citep{GriffithsLiederGoodman2015} therefore extends rational analysis to also take into account which cognitive operations are available to people, how long they take, and how costly they are. Given that resource-rational analysis has been successful at explaining a wide range of cognitive biases in judgment \citep{LiederGriffithsHuysGoodman2017a,LiederGriffithsHuysGoodman2017b} and decision-making \citep{LiederGriffithsHsu2017} by suggesting resource-efficient cognitive mechanisms, it might also be able to shed new light on other cognitive processes, such as planning.

%\paragraph{Specific Issue: How do people plan? Can human planning be understood in terms of resource-rational analysis}\fl{TODO: Fred and Falk}
Surprisingly little is known about how people plan. While there are several models of planning \citep{Newell1956,NewellSimon1972a,deGroot1965,Korf1987,Huys2012,Huys2015,Keramati2016} each of which explains some aspects of human planning in certain circumstances the exact mechanisms of human planning remain unclear; the applicability of each existing model is limited; and it remains unclear when people use which of those strategies and why. These questions are very difficult to answer because planning is an unobservable and highly complex cognitive process.

%\paragraph{Approach: Derive optimal planning strategies by meta-level RL and test their predictions using a novel process tracing paradigm}
Here, we address these problems by deriving planning strategies through resource-rational analysis and introducing a process tracing paradigm that makes people's planning strategies observable. We use the process data obtained with this paradigm to quantitatively evaluated our resource-rational model of planning against previously proposed planning strategies.

%\paragraph{Payoffs}
%\fl{TODO: Fred and Falk}
This approach enables us to automatically discover the optimal planning strategy for any given environment and thereby predict exactly how people's planning strategies should differ across different problems. Our newly developed process-tracing paradigm enables us to directly observe the sequence of people's planning operations -- allowing us to disambiguate process models of planning even when they predict the same final decision. We find that people's planning strategies are best explained by Model X, and systematically deviated from models Y, and Z. 

%\paragraph{Outline}
%\fl{TODO: Falk and Fred}
This paper is structured as follows. We start by introducing the methodology of resource-rational analysis and review previous findings on planning. Next, we introduce our new process-tracing paradigm for the study of planning and apply resource-rational analysis to its planning problems. We then evaluate the resource-rational model against process-tracing data from people in Experiment 1. Experiment 2 tests resource-rational predictions about how people's planning strategies should change with the structure of the environment. We close by discussing the implications of our findings for cognitive modeling and human rationality.

\section{Background}
\label{sec:background}

\subsection{Discovering optimal cognitive strategies}
%\fl{Here, we briefly introduce the method for discovering cognitive strategies by citing our previous work \cite{LiederKruegerGriffiths2017,LiederCallawayGulKruegerGriffiths2017}.}
Resource-rational analysis \cite{GriffithsLiederGoodman2015} derives process models how cognitive abilities are realized from a formal specification of their function and a model of the cognitive architecture available to realize them. Formally, the resource-rational model of a cognitive mechanism is defined as the solution to a constrained optimization problem over the space of strategies that can be implemented on the assumed cognitive architecture, and the objective function measures how well the strategy would perform in the limited time available to solve the problem. To compute resource-rational cognitive strategies defined in this way, we will apply the meta-level reinforcement learning method developed by \cite{LiederCallawayGulKruegerGriffiths2017} which generalizes and refines an earlier method for discovering rational heuristics for multi-alternative risky choice \citep{LiederKruegerGriffiths2017}. 
%\fl{TODO: Fred, Falk, Sayan, and Paul}

\subsection{Planning}
Historically, most research on planning has been conducted in the fields of problem solving and artificial intelligence \citep{NewellSimon1972a}. The Logic Theorist \citep{Newell1956}, which some consider the first artificially intelligent system planned its proofs using \textit{breadth-first search} -- first evaluating all possible one-step plans, then proceeding to all possible two-step plans, and so on, until a proof is discovered. By contrast, chess playing programs typically use \textit{depth-first search} -- evaluating one possible continuation in depth and then backing up one step at a time. Newell and Simon's General Problem Solver performed planned backward through a process called \textit{means-ends analysis} that compares a distant goal state to the current state to identify actions that could be taken to reduce the discrepancy between the two. By analyzing verbal protocols of people solving problems in crypt-arithmetic, logic, and chess, \cite{NewellSimon1972a} found that people usually plan forwards by a strategy called \textit{progressive deepening} \cite{deGroot1965}. Progressive deepening is similar to depth-first search in that starts by evaluating one potential solution in depth, but there are two important differences: First, progressive deepening starts again from the beginning after having considered a full sequence. Second, progressive deepening inspects all actions possible in the current state before looking deeper into one of them. More recent work has found that people often prune their decision tree when they encounter a large loss \citep{Huys2012}, cache and reuse previous action sequences \citep{Huys2015}, and use the implicit value estimate of their habit system to evaluate the intermediate states they consider during planning \citep{Keramati2016}.  Furthermore, it has been argued that people greedily choose each of their planning operations so as to maximize the immediate improvement in decision quality instead of considering the potential benefits of sequences of planning operations \citep{Gabaix2005}.
%\fl{TODO: Fred, Priyam, Falk, Paul, and Sayan}
